\documentclass{bioinfo}
\copyrightyear{2015}
\pubyear{2015}

\usepackage{amsmath}
\usepackage[ruled,vlined]{algorithm2e}
\usepackage{natbib}

\bibliographystyle{apalike}

\begin{document}
\firstpage{1}

\title[Long-read mapping and assembly]{Minimap and miniasm: fast mapping and de novo assembly for noisy long sequences}
\author[Li]{Heng Li}
\address{Broad Institute, 75 Ames Street, Cambridge, MA 02142, USA}
\maketitle

\begin{methods}

\section{Methods}

\subsection{Mapping}

\subsection{Assembly graph}

Let $\Sigma=\{\mathrm{A},\mathrm{C},\mathrm{G},\mathrm{T}\}$ be the
alphabet of nucleotides. If symbol $a\in\Sigma$, $\overline{a}$ is the
Watson-Crick complement of $a$. For a string $v=a_1a_2\cdots a_n$ over
$\Sigma$, which is also called a \emph{DNA sequence}, its \emph{reverse
complement} is $\overline{v}=\overline{a_1a_2\cdots
a_n}=\overline{a}_n\overline{a}_{n-1}\cdots\overline{a}_1$.

Two strings $v$ and $w$ may be mapped to each other based on their sequence
similarity. If $v$ can be mapped to a substring of $w$, we say $w$
\emph{contains} $v$. If a suffix of $v$ and a prefix of $w$ can be mapped to
each other, we say $v$ \emph{overlaps} $w$, or written as $v\to w$.
If we regard strings $v$ and $w$ as vertices, the overlap relationship defines
a directed edge between them. The \emph{length} of $v\to w$ equals the length
of $v$'s prefix that does not overlap $w$.

Let $G=(V,E,\ell)$ be a simple graph (i.e. no multi-edges or loops), where $V$ is a
set of DNA sequences (vertices), $E$ a set of overlaps between them (edges) and
$\ell:E\to\Re_+$ is the edge length fuction. $G$ is said to be
\emph{Watson-Crick complete} if i) $\forall v\in V$, $\overline{v}\in V$ and
ii) $\forall v\to w\in E$, $\overline{w}\to\overline{v}\in E$. $G$ is said to
be \emph{containment-free} if any sequence $v$ is not contained in other
sequences in $V$. If $G$ is both Watson-Crick complete and containment-free, it
is an \emph{assembly graph}.

\end{methods}

\end{document}
